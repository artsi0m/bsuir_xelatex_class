\subsection{Расчет теплового режима РЭС в перфорированном корпусе}

\begin{enumerate}[label={\arabic*.}]

\item Рассчитывается:
  поверхность корпуса блока,
  условная поверхность нагретой зоны,
  удельная мощность корпуса блока,
  удельная мощность нагретой зоны.

  $$S\mathrm{_{К}}=1,108\mathrm{м^2}$$
  $$S\mathrm{_{з}} = 0,655 \mathrm{м^2}$$
  $$q\mathrm{_к} = 13,53\mathrm{Вт/м^2}$$
  $$q\mathrm{_з} = 22,91 \mathrm{ Вт/м^2}$$

\item Находятся коэффициенты $\vartheta_1$ и
  $\vartheta_2$ в зависимости от
  удельной мощности корпуса блока и
  удельной мощности нагретой зоны.

  $$\vartheta_1=1,938$$
  $$\vartheta_2=3,182$$


\item Коэффициенты $K\mathrm{_{Н1}}$ и $K\mathrm{_{Н2}}$
  в зависимости от давления вне и
  внутри корпуса блока берутся из ГОСТ

  $$K\mathrm{_{Н1}} = 0,999$$
  $$K\mathrm{_{Н1}} = 0,996$$

\item Рассчитывается площадь (круглых)
  перфорационных отверстий~\cite{Rotkop1976}:
  \begin{equation}
    S = \frac{n \cdot \pi \cdot d^2}{4}
    \end{equation}
    Здесь n — количество отверстий $n = 60$,
    d – диаметр отверстия $d = 0,00635\mathrm{м}$.

    $$S = 0,0019\mathrm{м^2}$$

\item Рассчитывается коэффициент перфорации
    как отношение площади перфорации к
    площади основания корпуса ~\cite{Rotkop1976}:
    \begin{equation}
    \mathrm{П} = S\mathrm{_п}/2 l_1 l_2
  \end{equation}

  $$\mathrm{П} = 0,00702$$

\item Находится коэффициент $K\mathrm{_П}$ в зависимости от
  коэффициента перфораций:
  \begin{equation}
    K\mathrm{_П} = 0,29 + \frac{1}{1,41 + 4,95}
  \end{equation}
 
  $$K\mathrm{_П} = 0,982$$

\item Определяется прегрев корпуса блока~\cite{Rotkop1976}:
  \begin{equation}
    \vartheta\mathrm{_К} = \vartheta_1K\mathrm{_{Н1}}K\mathrm{_П} \cdot 0,93
  \end{equation}

  $$\vartheta\mathrm{_К} =0,012K$$

\item Определяется перегрев нагретой зоны~\cite{Rotkop1976}:
  \begin{equation}
\vartheta\mathrm{_з} = 0,93К\mathrm{_п}(\vartheta_1K\mathrm{_{Н1}} + (\vartheta_2\frac{1}{0,93} - \vartheta_1)K\mathrm{_{Н2}})
\end{equation}

$$\vartheta\mathrm{_з} = 3,246 K$$

\item Определяется средний перегрев воздуха вблоке~\cite{Rotkop1976}:
\begin{equation}
\vartheta\mathrm{_в} = \vartheta\mathrm{_з} \cdot 0,6
\end{equation}

$$\vartheta\mathrm{в} = 1,948 K$$


\item Рассчитывается удельная мощность элемента,
  перегрев поверхности элемента,
  перегрев окружающей элемент среды:

$$q\mathrm{_{эл}}= 73,934\mathrm{ВТ/м^2}$$

$$\vartheta\mathrm{_{эл}} = 5,05K$$

$$\vartheta\mathrm{_{эс}} = 3,03$$

\item Находятся температуры
  корпуса блока,
  нагретой зоны,
  поверхности элемента,
  воздуха в блоке,
  окружающей элемент среды:

  
$$T\mathrm{_К} = 313,012$$

$$T\mathrm{_з} = 316,246$$

$$T\mathrm{_{эл}} = 318,054$$

$$T\mathrm{_{в}} = 314,948$$

$$T\mathrm{_{эс}} = 316,032$$


\end{enumerate}
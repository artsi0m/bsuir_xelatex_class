\section{Расчет теплового режима РЭС при естественном воздушном охлаждении.}
\subsection{Расчет теплового режима РЭС в герметичном корпусе}
\begin{enumerate}[label={\arabic*.}]
  
\item Рассчитывается поверхность корпуса блока~\cite{Rotkop1976}:

  \begin{equation}
    S\mathrm{_{К}} = 2 \cdot (l_1 l_2 + (l_1+ l_2)l_3)
  \end{equation}

  $$S\mathrm{_{К}}=1,108\mathrm{м^2}$$

\item Определяется улсовная поверхность нагретой зоны ~\cite{Rotkop1976}:

  \begin{equation}
    S\mathrm{_{з}} = 2 (l_1 l_2 + (l_1 + l_2) K\mathrm{_{з}} l_3)
  \end{equation}

  $$S\mathrm{_{з}} = 0,366\mathrm{м^2}$$

\item Определяется удельная мощность корпуса по блоку ~\cite{Rotkop1976}:

\begin{equation}
  q\mathrm{_к} = P\mathrm{_з}/S\mathrm{_к}
\end{equation}

$$q\mathrm{_к} = 6,17\mathrm{Вт/м^2}$$

\item Рассчитывается удельная мощность нагретой зоны ~\cite{Rotkop1976}:
  \begin{equation}
      q\mathrm{_з} = P\mathrm{_з}/S\mathrm{_3}
    \end{equation}

    $$q\mathrm{_з} = 18,69 \mathrm{ Вт/м^2}$$

  \item Находится коэффициент $\vartheta_1$ в зависимости от удельной мощности корпуса блока ~\cite{Rotkop1976}:
    
\begin{equation}
\vartheta_1 = 0,1472q\mathrm{_к} - 0,2962 \cdot 10^{-3}q\mathrm{_к}^2 + 0,3127 \cdot 10^{-6}q\mathrm{_к}^2
\end{equation}

$$\vartheta_1=0,896$$

\item Находится коэффициент $\vartheta_2$ в зависимости от удельной мощности нагретой среды ~\cite{Rotkop1976}:

\begin{equation}
\vartheta_2 = 0,1390q\mathrm{_к} - 0,1223 \cdot 10^{-3}q\mathrm{_к}^2 + 0,0698 \cdot 10^{-6}q\mathrm{_з}^3
\end{equation}

$$\vartheta_2=2,556$$

\item Коэффициент $K\mathrm{_{Н1}}$ в зависмости от давления
  среды вне корпуса блока берётся из ГОСТ~\cite{GOST_15150-69}.

  $$K\mathrm{_{Н1}} = 0,999$$

  \item Коэффициент $K\mathrm{_{Н2}}$ в зависмости от давления
  среды внутри корпуса блока берётся из ГОСТ~\cite{GOST_15150-69}.

  $$K\mathrm{_{Н2}} = 0,999$$

\item Определяется перегрев корпуса блока ~\cite{Rotkop1976}:
  \begin{equation}
    \vartheta\mathrm{_к} = \vartheta_1 \cdot K\mathrm{_{Н1}}
  \end{equation}
  
  $$\vartheta\mathrm{_к} = 0,895 K$$

\item Рассчитывается перегрев нагретой зоны ~\cite{Rotkop1976}:
    \begin{equation}
    \vartheta\mathrm{_з} = \vartheta\mathrm{_к} + (\vartheta_2 - \vartheta_1) \cdot K\mathrm{_{H2}}
    \end{equation}

    $$\vartheta\mathrm{_з} = 0,525 K$$

  \item Определяется средний перегрев воздуха в блоке ~\cite{Rotkop1976}:
        \begin{equation}
      \vartheta\mathrm{_в} = 0,5 \cdot (\vartheta\mathrm{_к} + \vartheta\mathrm{_з})
    \end{equation}

    $$\vartheta\mathrm{_в} = 1,722 K$$

  \item Определяется удельная мощность элемента ~\cite{Rotkop1976}:
    \begin{equation}
      q\mathrm{_{эл}} = \frac{P_{эл}}{S_{эл}}
    \end{equation}

        $$q\mathrm{_{эл}} =144,172\mathrm{ВТ/м^2} $$

 \item Рассчитывается перегрев поверхности элемента ~\cite{Rotkop1976}:
 \begin{equation}
\vartheta\mathrm{_{эл}} = \vartheta\mathrm{_{з}}(a + b \frac{q\mathrm{_{Эл}}}{q\mathrm{_{з}}})
\end{equation}

$$\vartheta\mathrm{_{эл}} =6,825K$$

\item Рассчитывается перегрев окружающей элемент среды ~\cite{Rotkop1976}:

      \begin{equation}
      \vartheta\mathrm{_{эс}} = \vartheta\mathrm{_в}(0,75 + 0,25\frac{q\mathrm{_{эл}}}{q\mathrm{_{з}}})
    \end{equation}
    $$\vartheta\mathrm{_{эс}} = 4,611K$$

  \item Определяется температура корпуса блока ~\cite{Rotkop1976}:
    \begin{equation}
      T\mathrm{_к} = \vartheta\mathrm{_{к}} + T\mathrm{_с}
    \end{equation}
    $$T\mathrm{_{к}} = 313,895 K$$
    
\item Определяется температура нагретой зоны ~\cite{Rotkop1976}:
    \begin{equation}
      T\mathrm{_з} = \vartheta\mathrm{_з} + T\mathrm{_c}
    \end{equation}

    $$T\mathrm{_з} = 315,548 K$$
  \item Находится температура поверхности элемента ~\cite{Rotkop1976}:
    \begin{equation}
      T\mathrm{_{эл}} = \vartheta\mathrm{_{эл}} + T\mathrm{_c}
    \end{equation}

    $$T\mathrm{_{эл}} = 319,825 K$$

  \item Находится средняя температура воздуха в блоке ~\cite{Rotkop1976}:
    \begin{equation}
      T\mathrm{_{в}} = \vartheta\mathrm{_{в}} + T\mathrm{_c}
    \end{equation}

    $$T\mathrm{_{эл}} = 314,722 K$$

  \item Находится температура окружающей элемент среды ~\cite{Rotkop1976}:
    \begin{equation}
      T\mathrm{_{эс}} = \vartheta\mathrm{_{эс}} + T\mathrm{_c}
    \end{equation}

    $$T\mathrm{_{эc}} = 317,611 K$$
\end{enumerate}

\subsection{Определение адекватности полученных расчетных значений}

К сожалению, вынужден признать, что полученная модель
не достаточно адекватна оригиналу.

Такой вывод сделан на основании того, что в результате расчетов
получены значения, не соответствующие ожидаемым, а именно:

\begin{enumerate}[label={\arabic*.}]
\item При анализе корпуса блока самым эффективным оказалось
  не принудительное воздушное охлаждение,
  а охлаждение с исползованием перфорированного корпуса.
  
\item Не совпадения в некоторых случаях, значений
  температуры при внутреннем перемешивании в замкнутом корпусе
  с нулевой скоростью перемешивания и при использовании герметичного корпуса.
  
\end{enumerate}

Однако на основании результатов анализа теплового режима при обдуве,
использовании ребристого корпуса и перфорации можно сделать вывод,
что самым эффеткивным способом охлаждения будет:
использовани охлаждения обдувом корпуса.

Тем не менее маловероятно,
что данная РЭС будет охлаждаться именно таким способом,
поэтому остаётся только метод использования перфорации и ребристого корпуса.

И согласно анализу использование перфорации является самым
эффективным методом, после использование воздушного охлаждения.

В конце концов можно сделать вывод, что
использование в данной РЭС корпуса с перфорацией является,
если не лучшим, то достаточным компромиссным решением.

\newpage % Конец главы
\subsection{Определение адекватности полученных расчетных значений}

К сожалению, вынужден признать, что полученная модель
не достаточно адекватна оригиналу.

Такой вывод сделан на основании того, что в результате расчетов
получены значения, не соответствующие ожидаемым, а именно:

\begin{enumerate}[label={\arabic*.}]
\item Значения температуры при использовании оребрённого корпуса
оказались в несколько раз больше ожидаемых, а порой и самыми большими
среди представленных.

\item Значения температуры при использовании принудительного
воздушного охлаждения оказались близкими к значениям темпераутуры
окружающей среды.
  
\end{enumerate}

Тем не менее не беря в расчет температуру оребрённого корпуса можно
принять конструкторское решение между разными методами охлаждения, а
именно обдувом, использованием перфорированного корпуса и
принудительного воздушного охлаждения.

На основании проведённого анализа можно сделать вывод, что выбранный
конструкторами РЭС вариант охлаждения с использованием
перфорированного корпуса является самым адекватным
принципу работы устройства.

Конечно же использование принудительного охлаждения окажется,
ещё более эффективным, но оно не подходит для круглосуточно
работающего в жилом помещении устройства, с установленными для 
этого устройства требованиями по низкому энергопотреблению.

\newpage % Конец главы
\subsection{Расчет теплового режима РЭС в герметичном оребрённом
  корпусе}
\begin{enumerate}[label={\arabic*.}]

\item Рассчитывается поверхность неоребренного корпуса
  блока~\cite{Rotkop1976}: % (4.46)
  $$S\mathrm{_{К}}=1,108\mathrm{м^2}$$

\item Рассчитывается условная поверхность нагретой
  зоны~\cite{Rotkop1976}: % (4.39)
  $$S\mathrm{_{з}} = 0,655\mathrm{м^2}$$ 

\item Находится удельная мощность корпуса
  блока ~\cite{Rotkop1976}:  % (4.45)
  $$q\mathrm{_к} = 13,53\mathrm{Вт/м^2}$$

\item Находится удельная мощность нагретой
  зоны: % (4.38)
  $$q\mathrm{_з} = 22,91 \mathrm{ Вт/м^2}$$

\item Определяется коэффициент $\vartheta_1$ в зависимости от
  удельной мощности корпуса блока:

  $$\vartheta_1=1,938$$
\item Определяется коэффициент $\vartheta_2$ в зависимости от удельной мощности нагретой среды:
  $$\vartheta_2=3,182$$

\item Рассчитывается прегрев мужду нагретой зоной и корпусом
неоребренного блока

$$\vartheta_{21} = 1,24 K$$

\item Рассчитывается поверхность оребренного корпуса блока,
  как сумму поверхности блока и поверхности корпуса:
  \begin{equation}
    S\mathrm{_{кр}} = S\mathrm{_{кн}} + S_{p}
  \end{equation}

Суть добавления рёбер в данном случае заключается в увелечении площади
рассеивания. Возьмём площадь рёбер равной четверти от площади
корпуса. Тогда:
$$S\mathrm{_{кр}} = S\mathrm{_{кр}} \cdot 1,25 = 1,386$$

\item Рассчитаем удельную мощность оребренного
  корпуса блока~\cite{Rotkop1976}:
  \begin{equation}
      q\mathrm{_{кр}} = P\mathrm{_{з}} / S\mathrm{_{кр}}
    \end{equation}
    
    $$q\mathrm{_{кр}} = 10,826 \mathrm{Вт/м^2}$$

\item
    Определим коэффициент $\vartheta_{1p}$ в зависимости от удельной
    мощности оребрённого корпуса блока~\cite{Rotkop1976}:

\begin{equation}
\vartheta_{1p} = 0,1472q\mathrm{_{кр}} - 0,2962 \cdot 10^{-3}q\mathrm{_{кр}}^2 + 0,3127 \cdot 10^{-6}q\mathrm{_{кр}}^2      
\end{equation}

$$\vartheta_{1p}= 1,559K$$

\item Коэффициент $K\mathrm{_{Н1}}$ в зависмости от давления
  среды вне корпуса блока берётся из ГОСТ~\cite{GOST-15150-69}.

  $$K\mathrm{_{Н1}} = 0,999$$


\item Коэффициент $K\mathrm{_{Н2}}$ в зависмости от давления
  среды внутри корпуса блока берётся из ГОСТ~\cite{GOST-15150-69}.

  $$K\mathrm{_{Н2}} = 0,999$$

\item Рассчитывается перегрев оребренного корпуса блока
  \begin{equation}
    \vartheta\mathrm{_к} =\vartheta{_{1p}}K\mathrm{_{H1}}
  \end{equation}

  $$\vartheta\mathrm{_к} = 1,557K$$

\item Рассчитывается
  перегрев нагретой зоны
  с оребренным корпусом ~\cite{Rotkop1976}:

\begin{equation}
    \vartheta\mathrm{_з} = \vartheta{_к} +(\vartheta_2 - \vartheta_1)K\mathrm{_{Н2}}
  \end{equation}
  $$\vartheta\mathrm{_з} = 2,801K$$

\item Рассчитывается средний прогрев воздуха в блоке
  \begin{equation}
    \vartheta\mathrm{_в} = 0,75 \cdot \vartheta\mathrm{_з}
  \end{equation}
  
  $$\vartheta\mathrm{_в} = 2,1K$$

\item Определяется удельная мощность элемента, перегревы его
    поверхности и окружающей среды:
    $$q\mathrm{_{эл}} =73,93$$
    $$\vartheta\mathrm{_{эл}} = 53,87 K$$

    $$\vartheta\mathrm{_{эс}} = 4,359 K$$

  \item Находятся температуры поверхности корпуса блока, нагретой
зоны, поверхности элемента, воздуха в блоке и окружающей элемент
среды:
$$T\mathrm{_{к}} = 314,557 K$$
$$T\mathrm{_з} = 315,801 K$$
$$T\mathrm{_{эл}} = 315,100 K$$
$$T\mathrm{_{в}} = 366,875 K$$
$$T\mathrm{_{эс}} =317,36 K$$

\end{enumerate}
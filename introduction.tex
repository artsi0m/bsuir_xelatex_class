\begin{center}
\textbf{ВВЕДЕНИЕ}
\end{center}

\par
Беспроводной роутер ассиметричной цифровой абонентской линии — это
устройство предназначенное для предоставления доступа в интернет.
\par
Пользовательские устройства подключаются к роутеру по протоколу
беспроводной связи, а сам роутер включается в сеть интернет-провайдера
по модемной технологии ассиметричной цифровой абонентской линии.
Ассиметричность цифровой абонентской линии означает то, что исходящая
пропускная полоса канала связи может быть не равна входящей.
\par
В данном случае роутер предоставляет доступ в интернет в пределах
одного дома, квартиры или небольшого офиса.
Однако даже для того, чтобы обеспечить работу даже в пределах
небольшой домашней сети, требуются вычислительные ресурсы,
которые используются роутером для выполнения задач
по обеспечению работоспособности сети.
\par
В такие типовые задачи роутера входит:
\begin{itemize}[nosep]

\item Трансляция сетевых адресов;
\item Динамическая конфигурация подключаемых устройств;
\item Фильтрация сетевых пакетов.
\end{itemize}
В результате работы роутер нагревается, а охлаждение этого устройства
является основной темой данной курсовой работы.
\par
Цель данной курсовой работы — обосновать эффективность выбранной
системы охлаждения, для беспроводного роутера ассиметричной цифровой
абонентской линии 68.4 мВт, 12 В, модель ASW800 ADSL.
Задача курсовой работы – провести анализ теплового режима
радиоэлектронного средства в негерметичном перфарированным корпусом,
охлаждаемого с помощью естественной вентиляцией, и на основании
полученных результатов расчетов сделать вывод об эффективности
выбранного метода охлаждения.
\par
В курсовой работе использовались такие методы исследования, как:
\begin{itemize}[nosep]
\item аналитический;
\item физико-математический;
\item метод моделирования и компьютерной обработки данных.  
\end{itemize}
В первом разделе курсовой работы рассматривается и анализируется
устройство и его внутренние компоненты, по которым производится
расчёт теплового режима. Во втором разделе производятся расчеты
теплового режима РЭС и краткий анализ полученных данных.
В третьем разделе описывается процесс модерилования тепловой
картины поля микросхемы РЭС, описывается программный комплекс для моделирования.
В четвертом разделе сравниваются и анализируются данные, полученные при расчете
и определяется адекватность полученных данных.

\newpage
\section{Расчет теплового режима РЭС при естественном воздушном охлаждении.}
%% 4.2


%% 4.2.1


\begin{enumerate}[label={\arabic*.}]
\item Найдём коэффициенты
$\vartheta_1$ и $\vartheta_2$ зависящие
от удельной мощности корпуса и удельной мощности нагретой зоны.




$$\vartheta_2 = 0,525$$


\item Рассчитаем перегрев нагретой зоны $\vartheta\mathrm{_з}$,
  выбрав коэффициент $K\mathrm{_{Н2}}$ на основании данных из
  ГОСТ~\cite{GOST_15150-69}.
    $K\mathrm{_{Н2}} = 0,996$, 




  \item Определим удельную мощность элемента, используя данные о плошади
корпуса
    \begin{equation}
      q\mathrm{_{эл}} = \frac{P\mathrm{_{эл}}}{S\mathrm{_{кор}}}
    \end{equation}

    $$S\mathrm{кор} = 0,254\mathrm{м} \cdot 0,354\mathrm{м} = 0,09017\mathrm{м^2}$$
    $$q\mathrm{_{эл}} = \frac{13}{0,09017} =144,172\mathrm{ВТ/м^2} $$

  \item Определим перегрев поверхности элемента.
    \begin{equation}
      \vartheta\mathrm{_{эл}} = \vartheta\mathrm{_{з}}(a + b \frac{q\mathrm{_{Эл}}}{q\mathrm{_{з}}})
    \end{equation}

   $$\vartheta\mathrm{_{эл}} = 4,027K$$

  \item Рассчитаем перегрев окружающей элемент среды

    $$\vartheta\mathrm{_{эс}} = 3,02К$$
  \item Определим температуру корпуса блока
    \begin{equation}
      T\mathrm{_к} = \vartheta\mathrm{_к} + T\mathrm{_с}
    \end{equation}

    $$T\mathrm{_к} = 313,019K$$

  \item Определим температуру поверхности элемента
    \begin{equation}
      T\mathrm{_{Эл}} = \vartheta\mathrm{_{Эл}} + T\mathrm{_c}
    \end{equation}
    $$T\mathrm{_{Эл}} = 317,027 K$$

  \item Определим среднею температуру воздуха в блоке
    \begin{equation}
      T\mathrm{_в} = \vartheta\mathrm{_в} + T\mathrm{_c}
    \end{equation}

    $$T\mathrm{_в} =314,170 K$$
  \item Определим температуру окружающий элемент среды
    \begin{equation}
      T\mathrm{_{эс}} = \vartheta\mathrm{_{эс}} + T\mathrm{_c}
    \end{equation}
    
    $$T\mathrm{_{эс}} = 316,02 K$$
\end{enumerate}

\subsection{Расчет теплового режима РЭС в герметичном корпусе с внутренним перемешиванием}
%% 4.2.2.

\begin{enumerate}[label={\arabic*.}]
\item Найдём коэффициенты
$\vartheta_1$ и $\vartheta_2$ зависящие
от удельной мощности корпуса и удельной мощности нагретой зоны.

\begin{equation}
\vartheta_1 = 0,1472q\mathrm{_к} - 0,2962 \cdot 10^{-3}q\mathrm{_к}^2 + 0,3127 \cdot 10^{-6}q\mathrm{_к}^2
\end{equation}

$$\vartheta_1=0,0906$$\\

\begin{equation}
\vartheta_2 = 0,1390q\mathrm{_к} - 0,1223 \cdot 10^{-3}q\mathrm{_к}^2 + 0,0698 \cdot 10^{-6}q\mathrm{_з}^3
\end{equation}

$$\vartheta_2 = 0,525$$

\item Рассчитаем объём воздуха в корпусе
  \begin{equation}
    V\mathrm{_в} = l_1 l_2 l_3 (1 - K\mathrm{_з})
  \end{equation}

  $$V\mathrm{_в} = 0,687\mathrm{м^3}$$
  \item Поскольку в данном курсвом
    проекте рассмотрено \textbf{естественное} воздушное охлаждение.
    То разумно допустить, что вентилятор в корпусе отсуствует, т. е.
    скорость перемешивания воздуха в корпусе $W$ равно нулю.
    В обратном бы случае охлаждение считалось бы принудительным.

    $$W = 0$$
    
    \item Поскольку скорость перемешивания равна нулю, то зависящей от
      скорости пермешивания коэффицент $K_W$ также равен нулю.
      %% Сюда вставить этот график из matplotlib
    \item Найдем коэффицент $K_W$. Исходя из графика зависимости
коэффициента $K_W$ от скорости перемешивания, можно сделать вывод, что
при нулевой скорости перемешивания, коэффицент равен единице.
$$K_W = 1$$

\item Определим перегрев корпуса блока $\vartheta\mathrm{_к}$,
  выбрав коэффициент $K\mathrm{_{Н1}}$ на основании данных из
  ГОСТ~\cite{GOST_15150-69}.

  \begin{equation}
    \vartheta\mathrm{_к} = \vartheta_1 \cdot K\mathrm{_{Н1}}
  \end{equation}

  $K\mathrm{_{Н1}} = 0,999$, $\vartheta\mathrm{_к} = 0,905 K$


\item Формула же определения перегрева нагретой зоны отличается от той,
  что была в предыдущем подразделе:
  \begin{equation}
    \vartheta\mathrm{_з} = \vartheta_1(K\mathrm{_{Н1}} - 1) + \vartheta_2
    \end{equation}

    $$\vartheta\mathrm{_з}=0,525$$
  \item Определим средний перегрев воздуха в блоке.
    \begin{equation}
      \vartheta\mathrm{_в} = 0,75 \cdot \vartheta\mathrm{_з}
    \end{equation}
    $$\vartheta\mathrm{_в} = 0,0393$$
  \item Удельная мощность элемента найдена в предущем подразделе
    $q\mathrm{_{Эл}} =144,172\mathrm{ВТ/м^2} $

  \item Определим перегрев поверхности элемента.
    \begin{equation}
      \vartheta\mathrm{_{Эл}} = \vartheta\mathrm{_{з}}(a + b \frac{q\mathrm{_{ЭЛ}}}{q\mathrm{_{з}}})
    \end{equation}
    $$\vartheta\mathrm{_{Эл}} = 8,71$$
  \item Рассчитаем перегрев окружающей элемент среды
    \begin{equation}
      \vartheta\mathrm{_{эс}} = \vartheta\mathrm{_в}(0,75 + 0,25\frac{q\mathrm{_{эл}}}{q\mathrm{_{з}}})
    \end{equation}
    
    $$\vartheta\mathrm{_{эс}} = 6,54К$$
  \item Определим температуру корпуса блока
    \begin{equation}
      T\mathrm{_к} = \vartheta\mathrm{_к} + T\mathrm{_с}
    \end{equation}
    $T\mathrm{_к} = 314,963K$
      \item Определим температуру нагретой зоны
    \begin{equation}
      T\mathrm{_з} = \vartheta\mathrm{_з} + T\mathrm{_c}
    \end{equation}

    $$T\mathrm{_з} = 316,376 K$$

  \item Определим температуру поверхности элемента
    \begin{equation}
      T\mathrm{_{Эл}} = \vartheta\mathrm{_{Эл}} + T\mathrm{_c}
    \end{equation}
    $$T\mathrm{_{Эл}} = 321,716 K$$

  \item Определим среднею температуру воздуха в блоке
    \begin{equation}
      T\mathrm{_в} = \vartheta\mathrm{_в} +T\mathrm{_c}
    \end{equation}

    $$T\mathrm{_в} =315,532 K$$
  \item Определим температуру окружающий элемент среды
    \begin{equation}
      T\mathrm{_{эс}} = \vartheta\mathrm{_{эс}} + T\mathrm{_c}
    \end{equation}
    
    $$T\mathrm{_{эс}} = 319,537 K$$
    
\end{enumerate}
\subsection{Расчет теплового режима РЭС в герметичном корпусе с наружным обдувом}
%% 4.2.3.
\begin{enumerate}[label={\arabic*.}]
\item Найдём коэффициенты
$\vartheta_1$ и $\vartheta_2$ зависящие
от удельной мощности корпуса и удельной мощности нагретой зоны.

\begin{equation}
\vartheta_1 = 0,1472q\mathrm{_к} - 0,2962 \cdot 10^{-3}q\mathrm{_к}^2 + 0,3127 \cdot 10^{-6}q\mathrm{_к}^2
\end{equation}
$$\vartheta_1=0,0906K$$

\begin{equation}
\vartheta_2 = 0,1390q\mathrm{_к} - 0,1223 \cdot 10^{-3}q\mathrm{_к}^2 + 0,0698 \cdot 10^{-6}q\mathrm{_з}^3
\end{equation}

$$\vartheta_2 = 0,525K$$

\item Рассчитаем перегрев между нагретой зоной и корпусом блока
  \begin{equation}
    \vartheta_{21} = (\vartheta_2-\vartheta_1)K\mathrm{_{Н2}}
  \end{equation}
  $$\vartheta_{21}=0,43K$$

\item Рассчитаем перегрев корпуса блока с наружным обдувом
  \begin{equation}
    \vartheta\mathrm{_к} = q\mathrm{_к}/(12+4,17\nu)
  \end{equation}
  Здесь $\nu$ это скорость обдува. Примем её за 20 метров в секунду,
  основываясь на документации схожих устройств.
  $$\vartheta\mathrm{_к} = 0,0617/(12+4,17) = 0,000582K$$

\item Рассчитаем перегрев нагретой зоны блока с наружным обдувом
  \begin{equation}
    \vartheta\mathrm{_з} = \vartheta\mathrm{_к} + \vartheta_{21}
  \end{equation}
  $$\vartheta\mathrm{_з} = 0,044K$$
    \item Определим средний перегрев воздуха в блоке.
    \begin{equation}
      \vartheta\mathrm{_в} = 0,75 \cdot \vartheta\mathrm{_з}
    \end{equation}
    $$\vartheta\mathrm{_в} = 0,0307K$$

\item Определим удельную мощность элемента, используя данные о плошади
корпуса
    \begin{equation}
      q\mathrm{_{эл}} = \frac{P\mathrm{_{эл}}}{S\mathrm{_{кор}}}
    \end{equation}

    $$S\mathrm{кор} = 0,254\mathrm{м} \cdot 0,354\mathrm{м} = 0,09017\mathrm{м^2}$$
    $$q\mathrm{_{эл}} = \frac{13}{0,09017} =144,172\mathrm{ВТ/м^2} $$
  \item Определим перегрев поверхности элемента.
    \begin{equation}
      \vartheta\mathrm{_{эл}} = \vartheta\mathrm{_{з}}(a + b \frac{q\mathrm{_{Эл}}}{q\mathrm{_{з}}})
    \end{equation}

    $$\vartheta\mathrm{_{эл}} = 4,22$$

  \item Рассчитаем перегрев окружающей элемент среды
    \begin{equation}
      \vartheta\mathrm{_{эс}} = \vartheta\mathrm{_в}(0,75 + 0,25\frac{q\mathrm{_{эл}}}{q\mathrm{_{з}}})
    \end{equation}
    $$\vartheta\mathrm{_{эс}} = 3,163К$$

 \item Определим температуру корпуса блока
    \begin{equation}
      T\mathrm{_к} = \vartheta\mathrm{_к} + T\mathrm{_с}
    \end{equation}
    
    $$T\mathrm{_к} = 313,001K$$
  \item Определим температуру нагретой зоны
    \begin{equation}
      T\mathrm{_з} = \vartheta\mathrm{_з} + T\mathrm{_c}
    \end{equation}
    
    $$T\mathrm{_з} = 313,044 K$$

  \item Определим температуру поверхности элемента
    \begin{equation}
      T\mathrm{_{Эл}} = \vartheta\mathrm{_{Эл}} + T\mathrm{_c}
    \end{equation}
    $$T\mathrm{_{Эл}} = 317,217 K$$

  \item Определим среднею температуру воздуха в блоке
    \begin{equation}
      T\mathrm{_в} = \vartheta\mathrm{_в} + T\mathrm{_c}
    \end{equation}

    $$T\mathrm{_в} =314,170 K$$
  \item Определим температуру окружающий элемент среды
    \begin{equation}
      T\mathrm{_{эс}} = \vartheta\mathrm{_{эс}} + T\mathrm{_c}
    \end{equation}
    
    $$T\mathrm{_{эс}} = 316,163 K$$

\end{enumerate}

\subsection{Расчет теплового режима РЭС в герметичном оребрённом корпусе}
%% 4.2.4
\begin{enumerate}[label={\arabic*.}]
\item Найдём коэффициенты
$\vartheta_1$ и $\vartheta_2$ зависящие
от удельной мощности корпуса и удельной мощности нагретой зоны.

\begin{equation}
\vartheta_1 = 0,1472q\mathrm{_к} - 0,2962 \cdot 10^{-3}q\mathrm{_к}^2 + 0,3127 \cdot 10^{-6}q\mathrm{_к}^2
\end{equation}

$$\vartheta_1=0,0906K$$

\begin{equation}
\vartheta_2 = 0,1390q\mathrm{_к} - 0,1223 \cdot 10^{-3}q\mathrm{_к}^2 + 0,0698 \cdot 10^{-6}q\mathrm{_з}^3
\end{equation}

$$\vartheta_2 = 0,525K$$

\item Рассчитаем перегрев между нагретой зоной и корпусом оребренного
блока
\begin{equation}
  \vartheta_{21} = \vartheta{_2} - \vartheta{_1}
  \end{equation}

  $$\vartheta_{21} = 0,4343K$$

\item Рассчитаем поверхность оребренного корпуса блока,
  как сумму поверхности корпуса $S\mathrm{_{кр}}$
  и поверхности рёбер $S_{р}$.
  \begin{equation}
    S\vartheta{_{кр}} = S\vartheta{_{кн}} + S_{p}
  \end{equation}
  
  Суть добавления рёбер в данном случае заключается в увелечении площади
рассеивания. Возьмём площадь рёбер равной примерно четверти от площади
корпуса.
$$S\vartheta{_{кр}} = 1,38557\mathrm{м^2}$$
\item Рассчитаем удельную мощность корпуса блока
  
\begin{equation}
  q\mathrm{_{кр}} = P\mathrm{_{з}} / S\mathrm{_{кр}}
  \end{equation}

  $$q\mathrm{_{кр}} = 0,4937 \mathrm{Вт/м^2}$$
  
  \item Определим коэффициент $\vartheta_{1p}$ в зависимости от удельной
    мощности оребрённого корпуса

\begin{equation}
\vartheta_{1p} = 0,1472q\mathrm{_{кр}} - 0,2962 \cdot 10^{-3}q\mathrm{_{кр}}^2 + 0,3127 \cdot 10^{-6}q\mathrm{_{кр}}^2      
\end{equation}

$$\vartheta_{1p}= 0,0725K$$
\item Рассчитаем перегрев оребренного корпуса блока
  \begin{equation}
    \vartheta\mathrm{_к} =\vartheta{_{1p}}K\mathrm{_{H1}}
  \end{equation}

  $$\vartheta\mathrm{_к} = 0,0724K$$
\item Рассчитаем перегрев нагретой зоны с оребренным корпусом
  \begin{equation}
    \vartheta\mathrm{_з} = \vartheta{_к} +(\vartheta_2 - \vartheta_1)K\mathrm{_{Н2}}
  \end{equation}
  $$\vartheta\mathrm{_з} = 0,505K$$

\item Рассчитаем средний прогрев воздуха в блоке
  \begin{equation}
    \vartheta\mathrm{_в} = 0,75 \cdot \vartheta\mathrm{_з}
  \end{equation}
  $$\vartheta\mathrm{_в} = 0,3787K$$
    \item Определим перегрев поверхности элемента.
    \begin{equation}
      \vartheta\mathrm{_{эл}} = \vartheta\mathrm{_{з}}(a + b \frac{q\mathrm{_{Эл}}}{q\mathrm{_{з}}})
    \end{equation}\\

   $$\vartheta\mathrm{_{эл}} = 5,182K$$

  
  \item Рассчитаем перегрев окружающей элемент среды
    \begin{equation}
      \vartheta\mathrm{_{эс}} = \vartheta\mathrm{_в}(0,75 + 0,25\frac{q\mathrm{_{эл}}}{q\mathrm{_{з}}})
    \end{equation}

    $$\vartheta\mathrm{_{эс}} = 3,88К$$
      \item Определим температуру корпуса блока
    \begin{equation}
      T\mathrm{_к} = \vartheta\mathrm{_к} + T\mathrm{_с}
    \end{equation}

    $$T\mathrm{_к} = 313,072K$$
  \item Определим температуру нагретой зоны
    \begin{equation}
      T\mathrm{_з} = \vartheta\mathrm{_з} + T\mathrm{_c}
    \end{equation}
    $$T\mathrm{_з} = 313,505K$$

  \item Определим температуру поверхности элемента
    \begin{equation}
      T\mathrm{_{Эл}} = \vartheta\mathrm{_{Эл}} + T\mathrm{_c}
    \end{equation}
    $$T\mathrm{_{Эл}} = 318,183K$$

  \item Определим среднею температуру воздуха в блоке
    \begin{equation}
      T\mathrm{_в} = \vartheta\mathrm{_в} + T\mathrm{_c}
    \end{equation}

    $$T\mathrm{_в} =313,379 K$$
  \item Определим температуру окружающий элемент среды
    \begin{equation}
      T\mathrm{_{эс}} = \vartheta\mathrm{_{эс}} + T\mathrm{_c}
    \end{equation}
    
    $$T\mathrm{_{эс}} = 316,887 K$$


\end{enumerate}

\subsection{Расчет теплового режима РЭС в перфорированном корпусе}
%% 4.2.5
\begin{enumerate}[label={\arabic*.}]

\item Найдём коэффициенты
$\vartheta_1$ и $\vartheta_2$ зависящие
от удельной мощности корпуса и удельной мощности нагретой зоны.

\begin{equation}
\vartheta_1 = 0,1472q\mathrm{_к} - 0,2962 \cdot 10^{-3}q\mathrm{_к}^2 + 0,3127 \cdot 10^{-6}q\mathrm{_к}^2
\end{equation}

$$\vartheta_1=0,0906$$

\begin{equation}
\vartheta_2 = 0,1390q\mathrm{_к} - 0,1223 \cdot 10^{-3}q\mathrm{_к}^2 + 0,0698 \cdot 10^{-6}q\mathrm{_з}^3
\end{equation}

$$\vartheta_2 = 0,525$$

\item Рассчитаем площадь круглых перфорационных отверстий
  \begin{equation}
    S = \frac{n \cdot \pi \cdot d^2}{4}
  \end{equation}

  $$S=0,0012\mathrm{м^2}$$



\item Рассчитаем коэффициент перфорации
  \begin{equation}
    \mathrm{П} = S\mathrm{_п}/2 l_1 l_2
  \end{equation}
  $$\mathrm{П} = 0,00304$$

\item Находим коэффициент $K\mathrm{_П}$ зависящий от перфораций.
  $$K\mathrm{_П} \approx 1$$

\item Определяем перегрев корпуса блока
  \begin{equation}
    \vartheta\mathrm{_К} = \vartheta_1K\mathrm{_{Н1}}K\mathrm{_П} \cdot 0,93
  \end{equation}
  $$\vartheta\mathrm{_К} =0,0222K$$
\item Определяем перегрев нагретой зоны
  \begin{equation}
\vartheta\mathrm{_з} = 0,93К\mathrm{_п}(\vartheta_1K\mathrm{_{Н1}} + (\vartheta_2\frac{1}{0,93} - \vartheta_1)K\mathrm{_{Н2}})
  \end{equation}

  $$\vartheta\mathrm{_з} = 0,139K$$

\item Определяем средний перегрев воздуха в блоке
  \begin{equation}
\vartheta\mathrm{_в} = \vartheta\mathrm{_з} \cdot 0,6
\end{equation}
$$\vartheta\mathrm{в} = 0,0832$$

\item Рассчитаем удельную мощность элемента, перегрев поверхности
элемента, перегрев окружающей элемент среды по уже приведенным формулам
из предыдущих подразделов.
$$q\mathrm{_{эл}}= 144,172\mathrm{ВТ/м^2}$$

$$\vartheta\mathrm{_{з}} = 13,29$$

$$\vartheta\mathrm{_{эс}} = 7,98$$

\item Также, по уже приведённым формулам найдём температуры корпуса
блока, нагретой зоны, поверхности элемента, воздуха в блоке, окружающей
элемент среды.

$$T\mathrm{_К} = 313,022$$

$$T\mathrm{_з} = 313,083$$

$$T\mathrm{_{эл}} = 326,292$$

$$T\mathrm{_{в}} = 313,083$$

$$T\mathrm{_{эс}} = 320,975$$

\end{enumerate}

\subsection{Расчет теплового режима РЭС при принудительном охлаждении}%% 4.2.6

Для расчетов был выбран вентилятор от производитля \textit{5bites}
\textit{FB5010S-12L3} c диаметром лопастей 45мм.

\begin{enumerate}[label={\arabic*.}]
\item Определим средний перегрев воздуха в блоке
  \begin{equation}
    \vartheta\mathrm{_В} = 5 \cdot 10^{-4} \cdot \frac{P}{G}
  \end{equation}
Здесь G – воздушный поток, а P мощность рассеиваемая элементом.

$$\vartheta\mathrm{_В} = 0,109$$

\item Определим площадь поперечного в направлении продува сечения
корпуса блока (S):
\begin{equation}
  S = l_1 \cdot l_2
\end{equation}
$$S = 0,193538\mathrm{м^2}$$

\item Найдём коэффициент $m_1$ в зависимости от массового расхода
  охлаждаемого воздуха:
  \begin{equation}
    m_1 = 0,001G^{0,5} 
  \end{equation}
  $$m_1 = 0,003959$$

  \item Найдём коэффициент в зависомсти от поперечного в направлении
    продува сечения корпуса блока:
    \begin{equation}
      m_2 = (l_1 \cdot l_2) ^ {-0,406}
    \end{equation}
    $$m_2 = 1,948$$

    \item Найдём коэффициент $m_3$ в зависимости от длины корпуса блока
      в напралении продува:
      \begin{equation}
      m_3 = l_3 ^ {-1.059}= 2,994  
    \end{equation}
    $$m_3= 2,994$$

  \item Найдём коэффициент $m_4$ в зависимости от коэффициента заполнения:
    \begin{equation}
      m_4 = K\mathrm{_З}^{-0,42} \cdot (1 - K\mathrm{_з}^\frac{2}{3})
    \end{equation}
    $$m_4 = 35,713$$

  \item Рассчитаем перегрев нагретой зоны блока с принудительным охлаждением:
    \begin{equation}
      \vartheta\mathrm{_з} = \vartheta\mathrm{_в} + P m_1 m_2 m_3 m_4
    \end{equation}
    $$\vartheta\mathrm{_з} = 10,825K$$
  \item Найдём удельную мощность нагретой зоны:
    
\begin{equation}
  q\mathrm{_з} = P\mathrm{_з}/S\mathrm{_3}
\end{equation}

$$q\mathrm{_з} = 0,3789 \mathrm{Вт/м^2}$$

\item Рассчитаем перегрев поверхности элемента:
  \begin{equation}
    \vartheta\mathrm{_{эл}} = \vartheta\mathrm{_з}(0,75 + 0,25 \cdot \frac{q\mathrm{_{эл}}}{q\mathrm{_{з}}})(\frac{l_1}{l_3} + 0,5)
  \end{equation}
  $$\vartheta\mathrm{_{эл}} = 546,189K$$

  \item Рассчитаем перегрев окружающей элемент среды
  \begin{equation}
    \vartheta\mathrm{_{эс}} = \vartheta\mathrm{_в}(0,75 + 0,25 \cdot \frac{q\mathrm{_{эл}}}{q\mathrm{_{з}}})(\frac{l_1}{l_3} + 0,5)
  \end{equation}
  $$\vartheta\mathrm{_{эс}} = 5,143K$$

\item Определим температуру нагретой зоны:
  \begin{equation}
    T\mathrm{_з}  = \vartheta\mathrm{_з} + T\vartheta\mathrm{_c}
  \end{equation}
  $$T\mathrm{_з} = 323,825K$$

\item Определим температуру воздуха в блоке:
  \begin{equation}
    T\mathrm{_В} = \vartheta\mathrm{_в} + T\vartheta\mathrm{_c}
    \end{equation}

    $$T\mathrm{_В} = 313,102$$

  \item Определим температуру на выходе из блока:
    \begin{equation}
      T\mathrm{_{В2}} = 2\vartheta\mathrm{_В} + T\vartheta\mathrm{_C}
    \end{equation}
    $$T\mathrm{_{В2}} = 313,204$$

  \item Определим температуру поверхности элемента
    \begin{equation}
      T\mathrm{_{ЭЛ}} = \vartheta\mathrm{_{ЭЛ}} + T\vartheta\mathrm{_C}
    \end{equation}
    $$ T\mathrm{_{ЭЛ}} = 859,189 $$
    
  \item Определим тепмературу окружающей элемент воздуха
    \begin{equation}
      T\mathrm{_{ЭС}} = \vartheta\mathrm{_{ЭС}} + T\vartheta\mathrm{_C}
    \end{equation}
    $$T\mathrm{_{ЭС}} = 318,143$$
  \end{enumerate}

  \newpage
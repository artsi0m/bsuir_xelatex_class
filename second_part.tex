\section{Расчет теплового режима РЭС при естественном воздушном охлаждении.}
%% 4.2

Расчет теполового режима радиоэлектронных аппартов рекомендуется
проводить в три этапа~\cite{Rotkop1976}:
\begin{enumerate}[label={\arabic*.}]
  \item Определение среднеповерхностной температуры платы с
расположенными ней деталями, корпуса и температуры воздуха внтури
радиоэлектронного аппарата.
  \item Определить среднеповерхностные температуры корпусов элементов
  используя результаты первого этапа.
  \item Определить максимальные температуры критических зон элементов и
их функциональные связи со среднеповерхностной температурой как
корпусов, так и и плат.
\end{enumerate}

Первый и второй этапы расчета позволяют получить значения основных
параметров, связанных с выбором системы охлаждения, т.е. первых двух
этапов хватает для принятия конструкторсого решения касаемо выбора
системы охлаждения.

Полную систему уравнений теплообмена для реального аппарата часто
невозможно не только решить аналитически, но и строго записать. В
связи с этим процессы, протекающие в реальном радиоэлектронном
аппарате, схематизируют, принимают ряд упрощающик предпосылок и в
результате получают тепловую модель аппарата, для которой и проводят
рассчет теплового режима ~\cite{Rotkop1976}.

Наибольшее распространение получила весьма плодотворная схематичзация
процессов теплообмена в РЭА, предложенная Г.Н.Дульневым
~\cite{Dulnev1968}.

Суть метода заключается в том, что печатная плата с её элементами
принимается за одно тело с изотермической поверхностью (нагретую
зону), для которого и проводится расчет теплового режима.

Таким образом производится расчет среднеповерхностной температуры
нагретой зоны.

Под понятием нагретая зона понимается поверхность того элемента на
печатной плате, который рассеивает больше всего мощности.

В данном случае им является процессор ТNETD 7300.

В указанных ранее источниках нет инфомации о том каковы размеры это
чипа. Однако согласно буклету производителя данный чип принадлежит к
вычислительной архитектуре RISC MIPS 32 ~\cite{AR7_fact_sheet}.

Понимание того, к какой архитектуре относится процессор позволяет
определить чипсет, который им используется. В свою очередь данные о
чипсете позволяют сделать вывод о том какова площадь процессора.

Чипсет, размещаемый на материнской плате, выполняет функцию связующего
компонента (моста), обеспечивающего взаимодействие центрального
процессора (ЦП) c различными типами памяти, устройствами ввода-вывода,
контроллерами, как непосредственно через себя, так и через другие
контроллеры и адаптеры, с помощью многоуровневой системы
шин~\cite{Avdeev2019}.

Без чипсета процессор не сможет взаимодествовать с переферийными
устройствами напрямую.

Исходя из даты изготовления всей РЭС и архитекутры конкретного чипа,
можно сделать вывод, что процессор размещается на чипсете R8000
~\cite{R8000_physical_wikipedia}.

Согласно этим данным чипсет имеет форму прямоугольника со сторонами в
$l_1$ = 17,34 мм и $l_2$ = 17,30 мм (занимает площадь 299,98 мм$^2$) и
рассеивает 13 ват мощности.

Основываясь на информации о процессорах тех лет, примем высоту чипа
$l_3$ равной 2,5 мм ~\cite{MobilePentium3_wikipedia}.

Таким образом можно найти условную поверхность нагретой зоны по
формуле:

\begin{equation}
S \mathrm{_з} = 2 (l_1 l_2 + (l_1 + l_2) l_3 K \mathrm{_з} )
\end{equation}

\subsection{Расчет теплового режима РЭС в герметичном корпусе}
%% 4.2.1

\subsection{Расчет теплового режима РЭС в герметичном корпусе с внутренним перемешиванием}
%% 4.2.2.

\subsection{Расчет теплового режима РЭС в герметичном корпусе с наружным обдувом}
%% 4.2.3.

\subsection{Расчет теплового режима РЭС в герметичном оребрённом корпусе}
%% 4.2.4

\subsection{Расчет теплового режима РЭС в перфорированном корпусе}
%% 4.2.5

\subsection{Расчет теплового режима РЭС при принудительном охлаждении}%% 4.2.6
\section{ОБЩЕТЕХНИЧЕСКИЙ АНАЛИЗ ПРОЕКТИРУЕМОГО УСТРОЙСТВА}
\subsection{Анализ исходных данных}
\par
В курсовой работе рассматривается беспроводной роутер асимметричной
цифровой абонентской линии.  Чтобы кратко сформулировать назначение
данного сетевого устройства достаточно одного слова — маршрутизатор.
Потому что именно задачу маршрутизации, то есть доставки сетевых
пакетов из пользовательской сети в сеть интернет-провайдера решают
такого рода устройства.
\par
Одной из функций маршрутизатора является физическогое соединение
сетей. Маршрутизатор имеет несколько сетевых интерфейсов, подобных
интерфейсам компьютера, к каждому из которых может быть подключена
одна сеть. Маршрутизатор может быть реализован программно на базе
универсального компьютера (например, типовая конфигурация Unix или
Windows включает программный модуль маршрутизатора). Однако чаще
маршрутизаторы реализуются на базе специализированных аппаратных
платформ. В состав программного обеспечения маршрутизатора входят
протокольные модули сетевого уровня ~\cite{NetworksOlifer2016}.
\par
Именно такой специализированной аппартной платформой и является
рассматриваемое устройство. Чтобы ещё больше конкретизировать
назначение устройства необходимо упомянуть в каком сегменте сети оно
осуществляет свою работу.
\par
Локальная сеть (LAN, Local Area Network) — частная сеть,
функционирующая в отдельном здании и на прилегающей территории
(это может быть дом, офис или завод). LAN широко применяется для соединения персоналтьны компьютеров и бытовой электроники, позволяя совместно
использовать различные ресурсы (например, принтеры) и обмениваться
информацией ~\cite{NetworksTanenbaum2023}.

\subsection{Описание принципа работы анализируемого устройства}
\subsection{Анализ элементной базы устройства}
\subsection{Выбор и обоснование системы охлаждения}

\newpage
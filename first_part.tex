\section{ОБЩЕТЕХНИЧЕСКИЙ АНАЛИЗ ПРОЕКТИРУЕМОГО УСТРОЙСТВА}
\subsection{Анализ исходных данных}
\par
В курсовой работе рассматривается беспроводной роутер асимметричной
цифровой абонентской линии.  Чтобы кратко сформулировать назначение
данного сетевого устройства достаточно одного слова — маршрутизатор.
Потому что именно задачу маршрутизации, то есть доставки сетевых
пакетов из пользовательской сети в сеть интернет-провайдера решают
такого рода устройства.
\par
Одной из функций маршрутизатора является физическогое соединение
сетей. Маршрутизатор имеет несколько сетевых интерфейсов, подобных
интерфейсам компьютера, к каждому из которых может быть подключена
одна сеть. Маршрутизатор может быть реализован программно на базе
универсального компьютера (например, типовая конфигурация Unix или
Windows включает программный модуль маршрутизатора). Однако чаще
маршрутизаторы реализуются на базе специализированных аппаратных
платформ. В состав программного обеспечения маршрутизатора входят
протокольные модули сетевого уровня ~\cite{NetworksOlifer2016}.

Роутер выполнен в виде платы с распаянными компонентами, в
негерметичном корпусе с перфорацией для обеспечения естественной
вентиляции РЭС. Корпус имеет форму параллелепипида, на задней кромке
содержит разъём RJ-45 для подключения к телефонной сети и четыре
разъёма для подключения по стандарту Ethernet, антенну, разъём питания
и тумблер включения.

\includegraphics[scale = 0.5]{external_photos_back_view.png}

Схема электрическая принципиальная была взята с сайта федеральной
коммисии по связи США: \\
https://fccid.io/RK9-ASW800/Schematics/SCHEMATICS-478020 .

Поскольку в курсовой работе будет производиться расчёт тепловых
режимов будет также важно учитывать климатические факторы внешней
среды, то какие условия эксплутации при этом должны быть соблюдены
регламентирует соответсвующий стандарт — ГОСТ 15150-69.

Настоящий стандарт должен применяться при проектировании изделий.  В
частности, он должен применяться при состалвении технических заданий
на разработку или модернизацию изделий, а также при разработке
государственных стандартов и технических условий, устанавливающих
требования в части воздействия климатических факторов внешней среды
для группы изделий, а при отсуствии указанных групповых документов —
для отдельных видов изделий ~\cite{GOST_15150-69}.

Для конкретных типов или групп изделий виды воздействующих
климатических факторов и их номинальные значения устанавливаются в
зависимости от условий эксплуатации изделий в соответвующих
технических заданих, стандартах и технических условиях ~\cite{GOST_15150-69}.

Далее будут приведены данные о том, в каком именно сегменте сети
работает рассматриваемый роутер, но уже сейчас можно сделать какой
категории ГОСТ 15150-69 соответствует роутер, основываясь на его
положении в топологии сети. Дайнной категории соответсвует обозначение
УХЛ 4.2.

Характеристика данной категории следующая:\\
Для эксплуатации в помещниях (объемах) с искуственно регулируемыми
климатическими условиями, например в закрытых отапливаемых или
охлаждаемых и вентелируемых производственных и других, в том числе
хорошо вентилиуруемых подземных помещниях (отсуствие воздействия
прямого соленчного излучения, атмосферных осадков, ветра, песка и пыли
наружного воздуха; отсутвие или существенное уменьшение воздействия
рассеяного солнечного излучения и конденсации влаги). Для эксплуатации
в лабораторных, капитальных и других подобного типа помещениях ~\cite{GOST_15150-69}.

\subsection{Описание принципа работы анализируемого устройства}


Рассмотренной устройство — специализированная аппартная платформа,
реализующего фукнции маршрутизатора в сетевой топологии.  Чтобы ещё
больше конкретизировать назначение устройства необходимо упомянуть в
каком сегменте сети оно осуществляет свою работу.


Локальная сеть (LAN, Local Area Network) — частная сеть,
функционирующая в отдельном здании и на прилегающей территории
(это может быть дом, офис или завод). LAN широко применяется для соединения персоналтьны компьютеров и бытовой электроники, позволяя совместно
использовать различные ресурсы (например, принтеры) и обмениваться
информацией ~\cite{NetworksTanenbaum2023}.

На сегодняшний день беспроводные LAN применяются
повсеместно. Изначально они были популярны в жилых помещениях, старых
офисных зданиях, кафе и других местах, где прокладка кабелей обошлась
бы слишком дорого. В подобных система компьютеры обмениваются
информацией с помощью встроенного радиомодема и антенны. Чаще всего
компьютер обращается к специальному устройству, которой называется
точкой доступа (AP, Access Point), беспроводным маршрутизатором
(wireless router) или базовой станцией (base station). Это устройство
осуществляет передачу пакетов данных между беспроводными компьютерами,
а также между компьютером и интернетом. Точка доступа напоминает
популярного ребенка в школе, поскольку все хотят с ней «поговорить»
~\cite{NetworksTanenbaum2023}.

Одним из самых популярных стандартов беспроводных LAN является IEEE
802.11, более известный как wi-fi ~\cite{NetworksTanenbaum2023}.

И именно этому стандарту следует рассматриваемое устройство.

Вместо дорогостоящих лицензируемых частот системы 802.11 работают на
нелицензируемых полосах частот, например ISM («Industrial, Scientific,
and Medical» — «промышленные, научные и медицинские») устанавливаемых
МСЭ-R (например 902-928 МГц, 2,4-2,5 ГГц, 5,725-5,825 ГГц).  Этот
диапазон частот разрешено использовать любым устройствам, но мощность
их излучения должна быть ограничена, чтобы различные устройства не
мешали друг другу. Конечно, из-за этого 802.11-передатчики иногда
начинают конкурировать за частоты с беспроводными телефонами,
системами дистанционного открывания дверей гаража и микроволновками.
Так что до тех пор, пока пользователям не понадобиться позвонить
гаражным дверям, важно все настроить правильно ~\cite{NetworksTanenbaum2023}.



\subsection{Анализ элементной базы устройства}
\subsection{Выбор и обоснование системы охлаждения}

\newpage
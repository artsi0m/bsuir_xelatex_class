\subsection{Расчет теплового режима РЭС в герметичном корпусе с наружным обдувом}

\begin{enumerate}[label={\arabic*.}]
\item Рассчитывается поверхность корпуса блока~\cite{Rotkop1976}: % (4.46)
  $$S\mathrm{_{К}}=1,108\mathrm{м^2}$$

\item Рассчитывается условная поверхность нагретой зоны~\cite{Rotkop1976}: % (4.39)
  $$S\mathrm{_{з}} = 0,387\mathrm{м^2}$$ 
\item Находится удельная мощность корпуса блока ~\cite{Rotkop1976}:  % (4.45)
  $$q\mathrm{_к} = 13,53\mathrm{Вт/м^2}$$

\item Находится удельная мощность нагретой зоны ~\cite{Rotkop1976}: % (4.38)
  $$q\mathrm{_з} = 38,73 \mathrm{ Вт/м^2}$$

\item Определяется коэффициент $\vartheta_1$ в зависимости от удельной мощности корпуса блока ~\cite{Rotkop1976}:

  $$\vartheta_1=1,936$$
\item Определяется коэффициент $\vartheta_2$ в зависимости от удельной мощности нагретой среды ~\cite{Rotkop1976}:
  $$\vartheta_2=5,204$$
  

\item Коэффициент $K\mathrm{_{Н2}}$ в зависмости от давления
  среды вне корпуса блока берётся из ГОСТ~\cite{GOST_15150-69}.

  $$K\mathrm{_{Н1}} = 0,996$$

\item Рассчитывается перегрев между нагретой зоной и корпусом блока
  ~\cite{Rotkop1976}:
  \begin{equation}
    \vartheta_{21} = (\vartheta_{2}-\vartheta_{1})K\mathrm{_{Н2}}
    \end{equation}

    $$\vartheta_{21}=3,256 K$$

\item Рассчитывается перегрев корпуса блока с наружным обдувом
    ~\cite{Rotkop1976}:
    \begin{equation}
      \vartheta\mathrm{_к} = q\mathrm{_к}/(12 + 4,17 \nu)
      \end{equation}
      \nu — скорость обдува. Принятая равной $5\mathrm{м/с}$,
      как близкая к среднему значению скорости ветра бытового
вентилятора.

$$\vartheta\mathrm{_к} = 0,381$$

\item Определим прегрев корпуса блока с наружным обдувом
  ~\cite{Rotkop1976}:
  \begin{equation}
    \vartheta\mathrm{_з} =  \vartheta\mathrm{_к} + \vartheta_{21}
    \end{equation}
  
$$\vartheta\mathrm{_з} = 3,637$$

\item Определяется средний перегрев воздуха в блоке ~\cite{Rotkop1976}:
  \begin{equation}
    \vartheta\mathrm{_в} = 0,75 \cdot \vartheta\mathrm{_з}
  \end{equation}

  $$\vartheta\mathrm{_в} = 2,727$$
  \item Находится удельная мощность элемента~\cite{Rotkop1976}:
  $$q\mathrm{_{эл}} =221,803\mathrm{ВТ/м^2} $$
\item Рассчитывается перегрев поверхности элемента ~\cite{Rotkop1976}:
  $$\vartheta\mathrm{_{эл}} =7,934K$$

\item Рассчитывается перегрев окружающей элемент среды ~\cite{Rotkop1976}:
  $$\vartheta\mathrm{_{эс}} = 5,950K$$

  
  
\item Находится температуры корпуса блока, нагретой зоны, поверхности
  элемента, средняя температура в блоке и температура окружающей среды:
  $$T\mathrm{_{к}} = 313,381 K$$
  
    $$T\mathrm{_з} = 316,637 K$$
    $$T\mathrm{_{эл}} = 320,934 K$$
    $$T\mathrm{_{в}} = 315,728 K$$
    $$T\mathrm{_{эс}} =318,951$$

\end{enumerate}

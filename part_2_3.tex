\subsection{Расчет теплового режима РЭС в герметичном корпусе с наружным обдувом}

\begin{enumerate}[label={\arabic*.}]
\item Рассчитывается поверхность корпуса блока: % (4.46)
  $$S\mathrm{_{К}}=0,0424\mathrm{м^2}$$

\item Рассчитывается условная поверхность нагретой зоны: % (4.39)
  $$S\mathrm{_{з}} = 0,03088\mathrm{м^2}$$ 
\item Находится удельная мощность корпуса блока:  % (4.45)
  $$q\mathrm{_к} = 306,604\mathrm{Вт/м^2}$$

\item Находится удельная мощность нагретой зоны: % (4.38)
  $$q\mathrm{_з} = 420,05 \mathrm{ Вт/м^2}$$

\item Определяется коэффициент $\vartheta_1$ в зависимости от удельной мощности корпуса блока ~\cite{Rotkop1976}:

  $$\vartheta_1=17,32$$
\item Определяется коэффициент $\vartheta_2$ в зависимости от удельной мощности нагретой среды ~\cite{Rotkop1976}:
  $$\vartheta_2=42,05$$
  

\item Коэффициент $K\mathrm{_{Н2}}$ в зависмости от давления
  среды вне корпуса блока берётся из ГОСТ~\cite{GOST-15150-69}.

  $$K\mathrm{_{Н1}} = 0,996$$

\item Рассчитывается перегрев между нагретой зоной и корпусом блока
  ~\cite{Rotkop1976}:
  \begin{equation}
    \vartheta_{21} = (\vartheta_{2}-\vartheta_{1})K\mathrm{_{Н2}}
    \end{equation}

    $$\vartheta_{21}=24,73 K$$

\item Рассчитывается перегрев корпуса блока с наружным обдувом
    ~\cite{Rotkop1976}:
    \begin{equation}
      \vartheta\mathrm{_к} = q\mathrm{_к}/(12 + 4,17 \nu)
      \end{equation}
      \nu — скорость обдува. Принятая равной $5\mathrm{м/с}$,
      как близкая к среднему значению скорости ветра бытового
вентилятора.

$$\vartheta\mathrm{_к} = 9,33$$

\item Определим прегрев нагретой зоны блока с наружным обдувом
  ~\cite{Rotkop1976}:
  \begin{equation}
    \vartheta\mathrm{_з} =  \vartheta\mathrm{_к} + \vartheta_{21}
    \end{equation}
  
$$\vartheta\mathrm{_з} = 33,967$$

\item Определяется средний перегрев воздуха в блоке ~\cite{Rotkop1976}:
  \begin{equation}
    \vartheta\mathrm{_в} = 0,75 \cdot \vartheta\mathrm{_з}
  \end{equation}

  $$\vartheta\mathrm{_в} = 25,475$$
  \item Находится удельная мощность элемента~\cite{Rotkop1976}:
    $$q\mathrm{_{эл}} =1109,5096\mathrm{ВТ/м^2} $$
    
\item Рассчитывается перегрев поверхности элемента ~\cite{Rotkop1976}:
  $$\vartheta\mathrm{_{эл}} =47,855K$$

\item Рассчитывается перегрев окружающей элемент среды ~\cite{Rotkop1976}:
  $$\vartheta\mathrm{_{эс}} = 67,422K$$

\item Находится температуры корпуса блока, нагретой зоны, поверхности
  элемента, средняя температура в блоке и температура окружающей среды:
  $$T\mathrm{_{к}} = 322,333 K$$
  
    $$T\mathrm{_з} = 346,967 K$$
    $$T\mathrm{_{эл}} = 338,475 K$$
    $$T\mathrm{_{в}} = 360,855 K$$
    $$T\mathrm{_{эс}} =380,422 K$$

\end{enumerate}

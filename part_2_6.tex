\subsection{Расчет теплового режима РЭС при принудительном воздушном охлаждении}

Для расчетов был выбран вентилятор от производитля \textit{5bites}
\textit{FB5010S-12L3} c диаметром лопастей 45мм.

\begin{enumerate}[label={\arabic*.}]
\item Определяется средний перегрев воздуха в блоке
\begin{equation}
\vartheta\mathrm{_в} = 5 \cdot 10^{-4} \cdot \frac{P}{G}
\end{equation}

Здесь G – воздушный поток, а P мощность рассеиваемая элементом.

$$G = 0,06377\mathrm{м^3/c}$$

$$P = 20\mathrm{Вт}$$

$$\overline{\vartheta\mathrm{_в}} = 0,156K$$  
\item Определяется площадь поперечного
  в направлении продува сечения корпуса блока $$S= L_1 \cdot L_2$$,
  где $L_1$, $L_2$ — размеры корпуса блока,
  перпендикулярные направлению продува.
  $$S= L_1 \cdot L_2 =0,27\mathrm{м^2}$$

\item Находится коэффициент $m_1$ в зависимости
  от массового расхода охлаждающего воздуха ~\cite{Rotkop1976}:
  \begin{equation}
    m_1 = 0,001 \cdot G^{-0,5}
    \end{equation}
  $$m_1 = 0,0039$$

\item Находится коэффициент $m_2$,
  учитывающий величину площади поперечного
  к направлению обдува сечения аппарата\cite{Rotkop1976}:
  \begin{equation}
    m_2= (L_1 \cdot L_2) ^ {-0,406}
    \end{equation}

    $$m_2 = 1,7$$

  \item Находится коэффицент $m_3$,
    учитывающий длину аппарата в направлении обдува~\cite{Rotkop1976}:
    \begin{equation}
      m_3 = l_3 ^ {-1,059}
      \end{equation}

      $$m_3 = 4,269$$

\item Находится коэффициент $m_4$,
   учитывающий заполнение аппарата ~\cite{Rotkop1976}:
   \begin{equation}
     m_4 = K\mathrm{_{з}}^{-0,42}(1-K\mathrm{_{з}}^{2/3})^{0,5}
   \end{equation}
   $$m_4 = 1,595$$
  
\item Находится перегрев охлаждающего воздуха
  в радиоэлектронном аппарате ~\cite{Rotkop1976}:
  \begin{equation}
    \overline{\vartheta\mathrm{_{в}}} = 5 \cdot 10^{-4} P/G
  \end{equation}
  
  $$\overline{\vartheta\mathrm{_{в}}} = 0,157K$$

\item Рассчитывается перегрев нагретой зоны блока с принудительным

  охлаждением ~\cite{Rotkop1976}:
  \begin{equation}
    \vartheta\mathrm{_{з}} = \overline{\vartheta\mathrm{_{в}}} + P m_1 m_2 m_3 m_4
    \end{equation}
  

    $$\overline{\vartheta\mathrm{_{з}}} = 1,073 K$$

\item Находится условная поверхность нагретой зоны:
  $$S\mathrm{_{з}} = 0,654 \mathrm{м^3}$$

\item Находится удельная мощность нагретой зоны:
  $$q\mathrm{_{з}} = 22,1$$

\item Находится удельная мощность элемента:
  $$q\mathrm{_{эл}} = 73,93$$

\item Рассчитывается перегрев поверхности элемента ~\cite{Rotkop1976}
    \begin{equation}
    \vartheta\mathrm{_{эл}}   =  \vartheta\mathrm{_{з}} (0,75 + 0,25 \cdot q\mathrm{_{эл}} / q\mathrm{_{з}})(L /L_{3} + 0,5)
    \end{equation}

    $$\vartheta\mathrm{_{эл}} = 5,841$$

\item Рассчитывается перегрев окружающей
  среды у элемента ~\cite{Rotkop1976}:
  \begin{equation}
    \vartheta\mathrm{_{эс}} = \vartheta\mathrm{_{в}} (0,75 + 0,25 q\mathrm{_{эл}} / q\mathrm{_{з}}) ({L} / {L_{з}})
  \end{equation}
  
  $$\vartheta\mathrm{_{эс}} = 0,854 K$$

\item Определяется температура нагретой зоны ~\cite{Rotkop1976}:
  \begin{equation}
    T\mathrm{_{з}} = \vartheta\mathrm{_{з}} + T\mathrm{_{вх}}
  \end{equation}

  $$T\mathrm{_{з}} = 314,073K$$
  
\item Определяется средняя температура воздуха в блоке ~\cite{Rotkop1976}:
  \begin{equation}
    T\mathrm{_{в}} = \overline{\vartheta\mathrm{_{в}}} + T\mathrm{_{вх}}
  \end{equation}
  
  $$ T\mathrm{_{в}} = 313,157K$$
 
\item Определяется температура воздуха на выходе из
  блока ~\cite{Rotkop1976}:
  \begin{equation}
    T\mathrm{в2} = 2\cdot \overline{\vartheta\mathrm{_{в}}} + T\mathrm{_{вх}}
  \end{equation}
  
  $$ T\mathrm{в2} = 313,314K$$
  
\item Определяется температура поверхности элемента ~\cite{Rotkop1976}:
  \begin{equation}
    T\mathrm{_{эл}} = \vartheta\mathrm{_{эл}}  + T\mathrm{_{вх}}
  \end{equation}
  $$    T\mathrm{_{эл}} = 314,188K$$
  
\item Определяется температура окружающего
  элемент воздуха ~\cite{Rotkop1976}:
  \begin{equation}
        T\mathrm{_{эc}} = \vartheta\mathrm{_{c}} + T\mathrm{_{вх}}
    \end{equation}

 $$    T\mathrm{_{эл}} = 313,174K$$

  
\end{enumerate}
\newpage % Конец второй главы
